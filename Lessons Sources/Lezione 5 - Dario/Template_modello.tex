%% LyX 2.0.7 created this file.  For more info, see http://www.lyx.org/.
%% Do not edit unless you really know what you are doing.
\documentclass[twoside,english]{report}
\usepackage[sc]{mathpazo}
\usepackage{helvet}
\renewcommand{\ttdefault}{lmtt}
\usepackage[T1]{fontenc}
\usepackage[latin9]{inputenc}
\usepackage[a4paper]{geometry}
\geometry{verbose,lmargin=4cm,rmargin=2cm}
\usepackage{fancyhdr}
\pagestyle{fancy}
\setcounter{secnumdepth}{3}
\setcounter{tocdepth}{3}
\usepackage{babel}
\usepackage{float}
\usepackage{graphicx}
\usepackage[unicode=true,
 bookmarks=true,bookmarksnumbered=false,bookmarksopen=false,
 breaklinks=false,pdfborder={0 0 1},backref=false,colorlinks=false]
 {hyperref}

\makeatletter
%%%%%%%%%%%%%%%%%%%%%%%%%%%%%% User specified LaTeX commands.
% Customization file for the titlepage and document
%************************************************************
% Required stuff
%************************************************************
\usepackage{graphicx}
\usepackage{euler}
\usepackage[detect-all]{siunitx}
\usepackage{sectsty}
\allsectionsfont{\rmfamily}

% Page customization
\usepackage{fancyhdr}
\pagestyle{fancy}

% Color
\usepackage{color}
\definecolor{light-gray}{gray}{0.85}
\definecolor{dark-gray}{gray}{0.75}

\fancyhead{}  % clear all header fields
\fancyhead[LO,RE]{{\color{dark-gray}\fontsize{9}{11} \selectfont \myPhase}}
\fancyhead[RO,LE]{\includegraphics[height=0.7cm,keepaspectratio]{gfx/Skyward_logo}}
\fancyhead[CO,CE]{{\color{dark-gray}\fontsize{9}{11} \selectfont \myIPT}}
\fancyfoot{}  % clear all footer fields
\fancyfoot[RO,LE]{\fontsize{6}{11} \selectfont \includegraphics[height=0.2cm]{gfx/CC} This work is licensed under a Creative Commons Attribution-ShareAlike 4.0 International License.}
\fancyfoot[RE,LO]{\fontsize{9}{11} \selectfont \thepage}
\fancyheadoffset[LE,RO]{0.2pt}
\renewcommand{\headrulewidth}{0.2pt}
\renewcommand{\footrulewidth}{0.2pt}
\renewcommand{\headrule}{\hbox to\headwidth{%
   \leaders\hrule height \headrulewidth\hfill}}
\renewcommand{\footrule}{\hbox to\headwidth{%
    \leaders\hrule height \headrulewidth\hfill}}
\hypersetup{colorlinks=true, linkcolor=blue ,linktoc=page,citecolor=black}


%************************************************************
% Fancy stuff
%************************************************************
\newcommand{\titlecap}[1]{\huge{\textrm{#1}}}
\newcommand{\subtitlecap}[1]{\Large{\textsc{#1}}}
\newcommand{\sscap}[1]{\textbf{#1}}
\newcommand{\strong}[1]{\textbf{#1}}

%************************************************************
% Helpful stuff to modify here, not in the LyX Document
%************************************************************
\newcommand{\myDate}{\today}
\newcommand{\myGroup}{Skyward Experimental Rocketry}
\newcommand{\myUrl}{\url{http://www.skywarder.eu}}
\newcommand{\myUni}{Politecnico di Milano}

\newcommand{\myPhase}{Lesson II}
\newcommand{\myProject}{Aerospace System Guidance and Control}
\newcommand{\myIPT}{Aerospace System Guidance and Control}
\newcommand{\myTitle}{Dynamics and Kinematics}
\newcommand{\myAuthor}{Aureliano Rivolta}
\newcommand{\myEditor}{Francescodario Cuzzocrea}

\newcommand{\mail}[1]{\href{mailto:#1}{\texttt{#1}}}


\makeatother

\begin{document}
\input{Titlepage.tex}

%*******************************************************
% Titleback
%*******************************************************
\thispagestyle{empty}

\hfill
\vspace{5cm}

\strong{Abstract}\\
The first step in the control design process is to develop appropriate mathematical models of the system through physical laws or experimental data. In this section, we introduce the state-space and transfer function representations of dynamic systems.
Once appropriate mathematical models of a system have been obtained, either in state-space or transfer function form, we may then analyze these models to predict how the system will respond in both time and frequency domains. To put this in context, control systems are often designed to improve stability, speed of response, steady-state error, or prevent oscillations. In this section, we will show how to determine these dynamic properties from the system models.

\vfill


\medskip
\noindent{\sscap{Website}}: \\
\url{http://www.skywarder.eu}

\medskip
\noindent{\sscap{E-mail}}: \\
\mail{aureliano.rivolta@skywarder.eu}

\vspace{1cm}
\hrule
\bigskip
\clearpage


\pagenumbering{roman}

\tableofcontents{}

\clearpage{}

\pagenumbering{arabic}

\setcounter{page}{1}


\chapter{Introduction}


\section{What is Arduino ?}

Arduino is a single-board microcontroller inteded to make the application
of interactive objects or enviroment more accessible. 

So Arduino can be used for fast prototiping, or for building small
robots (or an airship, like in our case).

{*}Inserire immagine arduino{*}.

Pre-programmed into the on-board microcontroller chip is a boot loader
that allows uploading programs into the microcontroller memory without
needing a chip (device) programmer, simply by using the USB port.

Arduino is provided with input/output functionality so the board can
send/receive data to the external sensors.

The behaviour of the board is managed by microcontroller based on
the decisions implemented on the program that's running on the board.

The board may also interact with the external enviroment by using
actuator driven by the program trought the output channels.

{*}inserire immagine arduino collegato a qualche cosa che faccia molto
figo{*}


\section{What is Simulink ?}

Simulink is a data flow graphical programming language tool for modeling,
simulating and analyzing multidomain dynamic systestm.

The interface of Simulink consist in a graphical block diagramming
tool an customizable set of block libraries.

{*}immagine{*}

Plus, Simulink offer also integration with the rest of Matlab enviroment
an can either drive Matlab or can be called by a Matlab script.

Simulink can be enhanced with different kind of libraries, so we can
simulate the behaviour of a large variety of system and plus, Simulink
can be used to run our simulation on a targetted hardware board, in
this case for example we are going to use an Arduino Uno, but also
STM32 can be used (aggiungerne altre).

Penso di dover scrivere altro


\section{Downloading and installink Matlab and the Simulink libraries}

In order to use Simulink and the Arduino support package we first
need to install a reguar copy of MATLAB.

A free licence is available for all the students of Politecnico di
Milano.

In order to download it we first need to go to the following page
: 

http://www.software.polimi.it/software-download/studenti/matlab

Then we should click on ``attiva account''

{*}piazzare screen{*}

And follow the instruction on the monitor.

Once matlab and Simulink are installed, we should install the Arduino
Support Package for Simulink, so let's click on

inserire vari screen del procedimento

{*}pensavo di inserire altra roba di elettronica, ma devo chiedere
a mattia{*}


\section{A first simple example}

So let's start to play together with the Arduino Uno board and Simulink.

In this first example you will learn how to interface Arduino with
Simulink, so we can generate a firmware compatible with the board
starting from our model built and tested on Simulink.

First open up Simulink, simply by typing simulink from the MATLAB
command prompt :

\begin{figure}[H]
\noindent \begin{centering}
\includegraphics[scale=0.3]{\string"gfx/Start Simulink\string".jpg}
\par\end{centering}

\caption{Start Simulink}
\end{figure}


\newpage{}

So the Simulink library browser will show up : 

\begin{figure}[H]
\noindent \begin{centering}
\includegraphics[scale=0.3]{\string"gfx/Simulink Library Browser\string".jpg}
\par\end{centering}

\caption{Simulink Library Browser}
\end{figure}


For this first simple example we are going to use blocks from the
Simulink blockset and from the Simulink Support Package for Arduino
Hardware blockset.

Let's start a new blank project by clicking on File > New > Model.

We want to control the digital output of the Arduino Board in order
to made a led blink at specified time interval, so first we need to
select a Pulse Generator, simply by drag and dropping the Pulse Generator
block in the model window.

We need the Pulse Generator in order to generate the signal that we
will sent to the digital output of the Arduino Uno board : 

\begin{figure}[H]
\noindent \centering{}\includegraphics[scale=0.3]{\string"gfx/Pulse Generator\string".jpg}\caption{Pulse Generator}
\end{figure}


Then we need to connect the pulse generator to the digital outpu of
the Arduino Uno board. So let's browse the Simulink Support Package
for Arduino Hardware blockset and select the digital output block.

Drag and drop the block to the model and connect the pulse generator
to the block :

\begin{figure}[H]
\begin{centering}
\includegraphics[scale=0.3]{\string"gfx/Digital Output\string".jpg}
\par\end{centering}

\caption{Arduino Digital Output Block}
\end{figure}


In order to made our model work, we need to set the corret pulse type
of our pulse generator, so as our signal should control a digital
output, we need to set the ``sample based'' pulse type and set the
sample time to 0.1 (or wichever value you want to set) :

\begin{figure}[H]
\noindent \begin{centering}
\includegraphics[scale=0.3]{\string"gfx/Pulse Generator Options\string".jpg}
\par\end{centering}

\caption{Sample Generator Options}
\end{figure}


Now we are ready to test our model consistency with Simulink. Let's
connect a scope between the Pulse Generator block and the digital
output block, and next cick Run. 

Double clicking on the scope will show us the results of our simulation
: 

\begin{figure}[H]
\noindent \begin{centering}
\includegraphics[scale=0.3]{gfx/Scope}
\par\end{centering}

\caption{Simulation Results}
\end{figure}


Now we are ready to build the real circuit and to download the firmware
to the Arduino Uno board.

In the figure below is rapresented the wiring

\begin{figure}[H]
\noindent \begin{centering}
\includegraphics[scale=0.2]{\string"gfx/Primo esempio_bb\string".jpg}
\par\end{centering}

\caption{Wiring}
\end{figure}


So, after the wiring is done, we need to connect the Arduino Uno board
to our PC trought the USB cable.

On Simulink let's click on Tools > Run on Target Hardware > Prepare
to Run. 

A window will pop-up, we need to chose the Arduino Uno target hardware,
then click Apply. The Simulink defaults settings are perfectly suitable
for our needs

\begin{figure}[H]
\noindent \begin{centering}
\includegraphics[scale=0.3]{\string"gfx/Target Hardware\string".jpg}
\par\end{centering}

\caption{Target Hardware}
\end{figure}


Now click again on Tools > Run on Target Hardware > Run. 

Simulink will automatically generate the necessay C code based on
our model, and will flash the firmware on the board.

Take a look to the led, should blink.

{*}inserire immagine del led che blinka{*}


\section{Traffic Light Control}

In this example we will control three led in order to modelize a traffic
light flow. For this example Microsoft Visual C++ runtimes libraries
will be required


\section{Servo Control}

In this example we will learn how to use the Simulink Servo Control
blockset.

First thing, we need to 

\newpage{}AAAAAaaaaaaakdponfoeno fpjr rj

\newpage{}JKJNDJKBDKJNDJKB

\appendix

\chapter{Allora come va}

\newpage{}

Sksnbsubsuifbu
\end{document}
